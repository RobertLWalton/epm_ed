% File:		makepass.tex
% Author:	See Below
% Date:	        See Below
%
% The authors have placed this file in the public domain;
% they make no warranty and accept no liability for this file.

\documentclass[12pt]{article}
\usepackage{times}
\begin{document}
\newcommand{\problem}[1]{\underline{\Large \bf #1}}
\renewcommand{\section}[1]{\bigskip\underline{\bf #1}\\}
\newcommand{\header}[1]{\underline{\bf #1}}
\newcommand{\file}[1]{{\bf #1}}
\newcommand{\blankpage}{\newpage\vspace*{3.5in}%
    \centerline{\Large This Page is Intentionally Left Blank}}
\setlength{\parindent}{0.0in}
\setlength{\parskip}{1ex}

\problem{Make Passwords}

Ideally a password would be a string of 10 (or more)
randomly chosen decimal digits.  But this tends to be hard
to remember or type.

You are being asked to convert strings of decimal digits
into a more typeable and perhaps more memorable form using a 1-1 encoding.
In this encoding, groups of more than 3 digits are encoded
as 3-letter word randomly chosen from a dictionary.

Some examples are:\hspace{0.2in}
\begin{tabular}[t]{r@{~~~~~{\em is encoded as}~~~~~}l}
161954595198532 & 88bem06fis \\
174570539285673 & 5bid012loq \\
417744241130258 & dav034zeq9 \\
566373677809620 & sah9qub784 \\
270281313141987 & 4dud456niz \\
\end{tabular}

The encoding algorithm uses a dictionary of all words
of the form: \\
\hspace*{1in}$<consonant><vowel><consonant>$ \\
sorted in
lexical order, with {\tt y} treated as a consonant.
Dictionary word $0$ is {\tt "bab"} and word
$21*5*21-1=2204$ is {\tt "zuz"}.

Then the encoding algorithm with input number $N$ is as follows:
\begin{enumerate}
\item Divide $N$ by 6 and use the remainder to select the
encoding format thus:
\begin{tabular}{r@{~~~~~{\em selects}~~~~~}l}
0 & \tt "WWWDWWWDDD" \\
1 & \tt "WWWDDWWWDD" \\
2 & \tt "WWWDDDWWWD" \\
\end{tabular}
\hspace{0.3in}
\begin{tabular}{r@{~~~~~{\em selects}~~~~~}l}
3 & \tt "DWWWDDDWWW" \\
4 & \tt "DDWWWDDWWW" \\
5 & \tt "DDDWWWDWWW" \\
\end{tabular}
\item Process the format left to right.  If the next
character is {\tt D}, divide $N$ by 10 and output the
remainder.  If the next character is {\tt W}, divide $N$
by $21*5*21 = 2205$, use the remainder to look up
a word in the dictionary, output the word, and skip to
after the 3 {\tt W}'s in the format.
\end{enumerate}

For example, $0+6*(0+2205*(1+10*(2204+2205*(2+10*(3+10*4)))))= 126315290430$
is encoded as {\tt bab1zuz234}.  Note you must use `{\tt long}'
integers for C, C++, and JAVA, and that numbers input are treated
module $6*(21*5*21)^2*10^4=291721500000$.


\section{Input}
One more lines each containing a non-negative integer
with at most 12 digits.  Input ends with an end of line.


\section{Output}
For each input line, output one line containing the
encoded input integer.

\bigskip

\begin{center}
\begin{tabular}{ll}
\begin{minipage}[t]{2.5in}
\header{Sample Input}
\\[1ex]
\file{00-000-makepass.sin:}
\begin{verbatim}
0
6
13230
132300
291721500
2917215000
29172150000
\end{verbatim}
\file{00-001-makepass.sin:}
\begin{verbatim}
0
1
2
3
4
5
\end{verbatim}
\file{00-002-makepass.sin:}
\begin{verbatim}
126315290430
854595198532
870539285673
144241130258
273677809620
281313141987
\end{verbatim}
\end{minipage}
&
\begin{minipage}[t]{2.5in}
\header{Sample Output}
\\[1ex]
\file{00-000-makepass.sout:}
\begin{verbatim}
bab0bab000
bac0bab000
bab1bab000
bab0bac000
bab0bab100
bab0bab010
bab0bab001
\end{verbatim}
\file{00-001-makepass.sout:}
\begin{verbatim}
bab0bab000
bab00bab00
bab000bab0
0bab000bab
00bab00bab
000bab0bab
\end{verbatim}
\file{00-002-makepass.sout:}
\begin{verbatim}
bab1zuz234
88pat25yiq
5yuz930zok
vac975yuf4
lod3fan839
4qol723zeh
\end{verbatim}
\end{minipage}
\end{tabular}
\end{center}

\bigskip

\begin{tabular}{ll}
Author:	      & Robert L.~Walton $<$walton@acm.org$>$ \\
Date:         & Thu Oct  8 06:35:04 EDT 2020
\end{tabular}

The authors have placed this problem in the public domain;
they make no warranty and accept no liability for this problem.

\end{document}
